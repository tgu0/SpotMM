\documentclass{article}
\usepackage{amsmath}
\usepackage{amssymb}
\usepackage{algorithm}
\usepackage{algorithmic}

\title{Multi-Horizon Momentum Feature Engineering for High-Frequency Financial Data}
\author{}
\date{}

\begin{document}

\maketitle

\section{Overview}

We present a methodology for computing multi-scale momentum and volatility features from high-frequency financial time series data. The approach is designed to capture price dynamics across multiple temporal horizons while handling the irregularly-spaced nature of tick-level market data.

\section{Data Requirements}

The methodology assumes a time-indexed dataset with irregularly-spaced observations, where each observation $i$ has:
\begin{itemize}
    \item A timestamp $t_i$ (DateTime format)
    \item A volume-weighted average price (VWAP) denoted as $p_i^{VWAP}$
\end{itemize}

The input data is assumed to be chronologically ordered.

\section{Multi-Horizon Return Computation}

\subsection{Time-Lagged Price Matching}

For a set of temporal horizons $\mathcal{H} = \{10s, 30s, 60s, 300s\}$, we compute backward-looking returns using an asynchronous timestamp matching procedure. For each horizon $h \in \mathcal{H}$ and observation at time $t_i$, we:

\begin{enumerate}
    \item Define the target historical timestamp: $t_i^{(h)} = t_i - h$

    \item Identify the nearest available observation at or before $t_i^{(h)}$ within a tolerance window $\tau = h/2$

    \item Denote the matched historical price as $p_{i,h}^{lag}$, where:
    \begin{equation}
        p_{i,h}^{lag} = p_j^{VWAP} \quad \text{where} \quad j = \arg\max_{k: t_k \leq t_i^{(h)}, |t_k - t_i^{(h)}| \leq \tau} t_k
    \end{equation}
\end{enumerate}

This approach handles irregular sampling by finding the most recent historical observation within an acceptable tolerance, rather than assuming uniform time intervals.

\subsection{Return Calculation}

For each horizon $h$, the logarithmic return is computed as:
\begin{equation}
    r_{i,h} = \frac{p_i^{VWAP} - p_{i,h}^{lag}}{p_{i,h}^{lag}}
\end{equation}

This yields a set of returns at different time scales: $\{r_{i,10s}, r_{i,30s}, r_{i,60s}, r_{i,300s}\}$ for each observation $i$.

\section{Weighted Momentum Indicator}

To construct a composite momentum signal that emphasizes recent price movements, we compute a weighted linear combination:

\begin{equation}
    M_i^{weighted} = 0.5 \cdot r_{i,10s} + 0.3 \cdot r_{i,30s} + 0.2 \cdot r_{i,60s}
\end{equation}

The weights $(0.5, 0.3, 0.2)$ decay with increasing horizon length, giving higher importance to more recent price changes. This weighting scheme reflects the empirical observation that recent momentum tends to be more predictive of near-term price movements in high-frequency markets.

\section{Price Position Indicator}

To characterize the current price relative to its recent trading range, we compute a normalized price position over a 5-minute rolling window:

\begin{equation}
    \text{Position}_i^{5m} = \frac{p_i^{VWAP} - \min_{t_j \in [t_i - 5m, t_i]} p_j^{VWAP}}{\max_{t_j \in [t_i - 5m, t_i]} p_j^{VWAP} - \min_{t_j \in [t_i - 5m, t_i]} p_j^{VWAP}}
\end{equation}

This indicator takes values in $[0, 1]$, where:
\begin{itemize}
    \item $\text{Position}_i^{5m} \approx 0$ indicates the price is near the bottom of its recent range
    \item $\text{Position}_i^{5m} \approx 1$ indicates the price is near the top of its recent range
    \item $\text{Position}_i^{5m} \approx 0.5$ indicates the price is in the middle of its range
\end{itemize}

When the denominator equals zero (constant price over the window), the position is set to undefined (NaN).

\section{Multi-Scale Realized Volatility}

To capture the time-varying volatility at different frequencies, we compute realized volatility using the high-frequency 10-second returns as the base sampling frequency. For windows $W \in \{30s, 1m, 5m\}$:

\begin{equation}
    \sigma_i^W = \sqrt{\frac{1}{N_i^W - 1} \sum_{j: t_j \in [t_i - W, t_i]} (r_{j,10s} - \bar{r}_i^W)^2}
\end{equation}

where $N_i^W$ is the number of observations in the rolling window and $\bar{r}_i^W$ is the mean return over that window:

\begin{equation}
    \bar{r}_i^W = \frac{1}{N_i^W} \sum_{j: t_j \in [t_i - W, t_i]} r_{j,10s}
\end{equation}

This produces three volatility measures:
\begin{itemize}
    \item $\sigma_i^{30s}$: Ultra-short-term volatility
    \item $\sigma_i^{1m}$: Short-term volatility
    \item $\sigma_i^{5m}$: Medium-term volatility
\end{itemize}

The multi-scale approach allows for detection of volatility regime changes at different time scales, which is particularly important for risk management and position sizing in algorithmic trading.

\section{Output Features}

The complete feature set generated for each observation includes:

\begin{table}[h]
\centering
\begin{tabular}{ll}
\hline
\textbf{Feature} & \textbf{Description} \\
\hline
$r_{i,10s}$ & 10-second return \\
$r_{i,30s}$ & 30-second return \\
$r_{i,60s}$ & 60-second return \\
$r_{i,300s}$ & 300-second (5-minute) return \\
$M_i^{weighted}$ & Weighted momentum composite \\
$\text{Position}_i^{5m}$ & Normalized price position (5-min window) \\
$\sigma_i^{30s}$ & 30-second realized volatility \\
$\sigma_i^{1m}$ & 1-minute realized volatility \\
$\sigma_i^{5m}$ & 5-minute realized volatility \\
\hline
\end{tabular}
\caption{Complete set of momentum and volatility features}
\end{table}

\section{Handling Missing Data}

Several mechanisms ensure robustness to missing or irregular data:

\begin{enumerate}
    \item \textbf{Timestamp matching tolerance}: When looking back for historical prices, a tolerance of $h/2$ ensures that matches are only made when sufficiently recent data is available

    \item \textbf{Undefined returns}: When no valid historical price is found within the tolerance window, the return is set to NaN

    \item \textbf{Rolling window validity}: Volatility and price position indicators require sufficient observations within their respective windows; otherwise they evaluate to NaN

    \item \textbf{Division by zero protection}: When computing normalized metrics, zero denominators result in NaN values rather than infinite values
\end{enumerate}

\section{Computational Considerations}

The methodology is designed for efficient computation on large-scale tick data:

\begin{itemize}
    \item The asynchronous merge operation (merge-as-of) has time complexity $O(n \log n)$ for $n$ observations
    \item Rolling window computations are performed using time-based windows rather than fixed counts, accommodating irregular sampling
    \item All operations are vectorized over the entire dataset, avoiding explicit loops over observations
\end{itemize}

\section{Applications}

These features are particularly suited for:
\begin{itemize}
    \item Short-term price prediction models
    \item Market microstructure analysis
    \item High-frequency trading signal generation
    \item Volatility forecasting
    \item Regime detection and classification
\end{itemize}

The multi-scale nature of the features allows machine learning models to learn relationships between price dynamics at different time horizons, capturing both momentum persistence and mean reversion effects.

\end{document}
